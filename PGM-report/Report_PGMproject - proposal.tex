%%% D\'ebut du pr\'eambule %%%

%%%%%%%%%%%%%%%%%%%%%%%%%%%%%%%%%
% Options g\'en\'erales du document %
%%%%%%%%%%%%%%%%%%%%%%%%%%%%%%%%%

\documentclass[12pt]{report}	% Classe du document

%%%%%%%%%%%
% Langues %
%%%%%%%%%%%

\usepackage[english]{babel}		% Francisation de LaTeX

%%%%%%%%%%%%%%%%%%%%%%%
% Encodage, fontes... %
%%%%%%%%%%%%%%%%%%%%%%%


\usepackage[T1]{fontenc}
\usepackage[latin9]{inputenc}
\usepackage{amssymb}

\usepackage{latexsym}
\usepackage{amsmath,bm}
%\usepackage{graphicx}
\usepackage{aeguill}
\usepackage{aecompl}
\usepackage{url}
\usepackage{scribe_MG}
\usepackage{float}
\usepackage{graphicx}
\usepackage{caption}
\usepackage{subcaption}
\usepackage{MnSymbol}
%\usepackage[lite]{mtpro2}
 		% Encodage d'entr\'ee : permet l'utilisation de caract\`eres accentu\'es en entr\'ee


\usepackage{graphicx}			% pour les images
\usepackage{float}
%%% Fin du pr\'eambule %%%
\def\ts{\top}
\def\XX{\mathbf{X}}
\def\wb{\mathbf{w}}
\def\xb{\mathbf{x}}
\def\yb{\mathbf{y}}
\def\db{\mathbf{d}}
\def\hb{\mathbf{h}}
\def\Diag{\text{Diag}}

\def\etab{\boldsymbol \eta}

%%% D\'ebut du document %%%

\begin{document}
	 
\scribe{}
%%\lecturenumber{2}			% required, must be a number
\hwnumber{1}			% required, must be a 
%%\lecturedate{October 10}		% required, omit year
\hwdate{November 25, 2014}		% required, omit year

	
\maketitle

%	\underline{For your information}\\
%\BIT
%\item
%Email: 
%\url{shuyu.dong@polytechnique.edu}
%\EIT

\begin{center}
\textbf{Joint project with Object recognition and computer vision (RecVis Topic 1.B) - Joint representations for images and text - }
\end{center}

\paragraph{Main objectif --- Multimodal retrival:} image-to-image search, tag-to-image search, and image-to-tag search. \\

\textbf{Objectif of implementations}: building a retrival pipeline using canonical correlation analysis.

\paragraph{Work plan:} 
%\subsection{2. Datasets and features} 
\begin{itemize}
\item Preparation for implementations: generalities on \textit{word2vec}, \textit{Overfeat} and/or \textit{matconvnet} projects.
\begin{itemize}
\item word2vec: ready for feature extraction.
\item Overfeat: extracted certain features, but don't know if it works right: there are minor errors around \texttt{convert} and jpeg delegate library(to be rectified if possible).
\item matconvnet: ready for feature extraction.
\end{itemize} 
\item Theory studies on graphical models around CCA: according to the paper [1], from two-view CCA to three-view CCA with the third view from semantic/ground truth features.

\begin{itemize}
\item[-] In case of complete absence of ground truth category for some images: recovering absente topics by cluster-based information retrival on tag features.
\end{itemize}
\item Feature extraction: CNN features using \textit{Overfeat} or \textit{matconvnet}; text features using \textit{word2vec}.
\begin{itemize}
\item[-] Bibliographic search on related works: features that are used in the article[1](besides CNN features) and in other related works.
\item[-] Experiments on the pertinence of similarity measures used in references. Find potential mistakes/problems in feature extraction.
\end{itemize}
\item Implemente a retrival system using the two-view and three-view CCAs based on the computed features; compare the performances. 

\end{itemize}








\end{document}
%%% Fin du document %%%
	
